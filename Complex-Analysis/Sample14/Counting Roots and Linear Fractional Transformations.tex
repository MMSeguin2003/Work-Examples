\documentclass{article}
\usepackage{graphicx} % Required for inserting images
\usepackage[utf8]{inputenc}
\usepackage{setspace}
\usepackage[margin=1.5cm]{geometry}
\usepackage{amsmath}
\usepackage{amsthm}
\usepackage{amsfonts}
\usepackage{indentfirst}

\title{Counting Roots and Linear Fractional Transformations}
\author{Matthew Seguin}
\date{}

\begin{document}

\maketitle

\section*{94.2}
\begin{center}
    \doublespacing
    Let $f$ be a function that is analytic inside and on a positively oriented simple closed contour $C$.
    \\Further suppose that $f$ is never zero on $C$.
    \\Let the image of $C$ under $w = f(z)$ be the closed contour $\Gamma$ as shown in the book.
    \\If you imagine a person standing at the origin ($w = 0$), initially looking to any point $w_0\in\Gamma$ and tracing the path of someone else walking around $\Gamma$, then that person at the origin will do 3 complete counterclockwise rotations while tracing the other's path.
    \\Therefore we know the winding number is 3 and so $\frac{1}{2\pi}\Delta _C\:arg\:f(z) = 3$ which gives the result $\Delta _C\:arg\:f(z) = 6\pi$.
    \\Since $f$ is analytic inside and on $C$ we know it has 0 poles inside $C$ (that is $N_{\infty} = 0$).
    \\Furthermore since $f$ is nonzero on $C$ we know that $\frac{1}{2\pi}\Delta _C\:arg\:f(z) = 3 = N_0 - N_{\infty}$ where $N_0$ is the number of zeros of $f$ inside $C$ and $N_{\infty}$ is the number of poles of $f$ inside $C$ (which we know is 0).
    \\Therefore we have that $f$ has 3 zeros (counting multiplicity) interior to $C$ and $\Delta _C\:arg\:f(z) = 6\pi$ \qedsymbol
    \break
\end{center}


\newpage
\section*{94.6}
\begin{center}
    Let $C$ be the unit circle centered at the origin $|z| = 1$, clearly $C$ is a simple closed contour.
\end{center}

{\Large\textbf{a.}}
\begin{center}
    \doublespacing
    Clearly $f(z) = -5z^4$ and $g(z) = z^6 + z^3 - 2z$ are both analytic inside and on $C$.
    \\Furthermore we know that for all $z\in C$:
    \\$|g(z)| = |z^6 + z^3 - 2z|\leq |z^6| + |z^3| + |-2z| = |z|^6 + |z|^3 + 2|z| = 4 < 5 = 5|z|^4 = |-5z^4| =  |f(z)|$
    \\Therefore by Rouche's Theorem we know $f(z) = -5z^4$ and $f(z) + g(z) = z^6 - 5z^4 + z^3 - 2z$ have the same number of zeros (counting multiplicity) inside $C$.
    \\Clearly $f(z) = -5z^4$ has 4 zeros (counting multiplicity) inside $C$ so we know $f(z) + g(z) = z^6 - 5z^4 + z^3 - 2z$ has 4 zeros inside $C$ \qedsymbol
\end{center}

{\Large\textbf{b.}}
\begin{center}
    \doublespacing
    Clearly $f(z) = 9$ and $g(z) = 2z^4 - 2z^3 + 2z^2 - 2z$ are both analytic inside and on $C$.
    \\Furthermore we know that for all $z\in C$:
    \\$|g(z)| = |2z^4 - 2z^3 + 2z^2 - 2z|\leq |2z^4| + |-2z^3| + |2z^2| + |-2z| = 2|z|^4 + 2|z|^3 + 2|z|^2 + 2|z| = 8 < 9 = |f(z)|$
    \\Therefore by Rouche's Theorem we know $f(z) = 9$ and $f(z) + g(z) = 2z^4 - 2z^3 + 2z^2 - 2z + 9$ have the same number of zeros (counting multiplicity) inside $C$.
    \\Clearly $f(z) = 9$ has no zeros inside $C$ so we know $f(z) + g(z) = 2z^4 - 2z^3 + 2z^2 - 2z + 9$ has no zeros inside $C$ \qedsymbol
\end{center}

{\Large\textbf{c.}}
\begin{center}
    \doublespacing
    Clearly $f(z) = -4z^3$ and $g(z) = z^7 + z - 1$ are both analytic inside and on $C$.
    \\Furthermore we know that for all $z\in C$:
    \\$|g(z)| = |z^7 + z - 1|\leq |z^7| + |z| + |-1| = |z|^7 + |z| + 1 = 3 < 4 = 4|z|^3 = |-4z^3| = |f(z)|$
    \\Therefore by Rouche's Theorem we know $f(z) = -4z^3$ and $f(z) + g(z) = z^7 - 4z^3 + z - 1$ have the same number of zeros (counting multiplicity) inside $C$.
    \\Clearly $f(z) = -4z^3$ has 3 zeros (counting multiplicity) inside $C$ so we know $f(z) + g(z) = z^7 - 4z^3 + z - 1$ has 3 zeros inside $C$ \qedsymbol
\end{center}


\newpage
\section*{94.9}
\begin{center}
    \doublespacing
    Let $c\in\mathbb{C}$ such that $|c| > e$ then consider the functions $f(z) = cz^n$ for some $n\in\mathbb{N}$ and $g(z) = -e^z$.
    \\Let $C$ be the unit circle $|z| = 1$. Clearly both are analytic inside and on $C$.
    \\On $C$ we know $|f(z)| = |cz^n| = |c||z|^n = |c| > e$.
    \\We also know if $z = x + iy$ then $|g(z)| = |-e^z| = |-e^{x+iy}| = |-e^x||e^{iy}| = e^x$.
    \\On $C$ this is clearly maximized my maximizing $x =Re\:z$ and hence is maximized at $z = 1$.
    \\So on $C$ we have that $|f(z)| = |c| > e = |-e^1|\geq |e^z| = |g(z)|$.
    \\Therefore we know $f(z) = cz^n$ and $f(z) + g(z) = cz^n - e^z$ have the same number of zeros (counting multiplicity) inside $C$.
    \\Clearly $f(z) = cz^n$ has $n$ zeros (counting multiplicity) inside $C$ (namely all at $z = 0$), so we know $f(z) + g(z) = cz^n - e^z$ has $n$ zeros inside $C$.
    \\This is equivalent to saying the equation $cz^n = e^z$ has $n$ roots inside $C$ \qedsymbol
\end{center}


\newpage
\section*{98.12}
\begin{center}
    \doublespacing
    Consider the circle $|z - z_0| = R$ under the map $f(z) =\frac{1}{z} =\frac{\overline{z}}{|z|^2}$.
    \\Given we know that the circle is mapped to another circle we know that neither circle passes through the origin.
    \\Recall that we may represent such a circle with the equation:
    \[A(x^2 + y^2) + Bx + Cy + D = 0\]
    Where $A\neq 0$ because it's a circle and $D\neq 0$ because it does not pass through the origin.
    \\This is also equivalent to:
    \[\Big{(}x +\frac{B}{2A}\Big{)}^2 +\Big{(}y +\frac{C}{2A}\Big{)}^2 =\Big{(}\frac{\sqrt{B^2 + C^2 - 4AD}}{2A}\Big{)}^2\]
    Implying that the such a circle has center $z_0 = -\frac{B}{2A} - i\frac{C}{2A}$ and radius $R =\Big{(}\frac{\sqrt{B^2 + C^2 - 4AD}}{2A}\Big{)}^2$
    \\Further recall that this under $f(z) =\frac{1}{z} = u(x, y) + iv(x, y)$ is mapped to:
    \[D(u^2 + v^2) + Bu - Cv + A = 0\]
    We may also write this as:
    \[u^2 +\frac{B}{D} u + v^2 -\frac{C}{D} v +\frac{A}{D} = 0\]
    \[u^2 + 2\frac{B}{2D} u + (\frac{B}{2D})^2 + v^2 - 2\frac{C}{2D} + (-\frac{C}{2D})^2 = -\frac{A}{D} + (\frac{B}{2D})^2 + (-\frac{C}{2D})^2\]
    \[\Big{(}u +\frac{B}{2D}\Big{)}^2 +\Big{(}v -\frac{C}{2D}\Big{)}^2 =\frac{B^2 + C^2 - 4AD}{4D^2} =\Big{(}\frac{\sqrt{B^2 + C^2 - 4AD}}{2D}\Big{)}^2\]
    This is clearly the equation of a circle with center $w_0 = -\frac{B}{2D} + i\frac{C}{2D}$ and radius $\rho =\Big{(}\frac{\sqrt{B^2 + C^2 - 4AD}}{2D}\Big{)}^2$.
    \\If we had $z_0 = w_0$ then it must be that $-\frac{B}{2A} - i\frac{C}{2A} = -\frac{B}{2D} + i\frac{C}{2D}$.
    \\Giving the simultaneous equations $-\frac{B}{2A} = -\frac{B}{2D}$ (hence $A = D$) and $-\frac{C}{2A} =\frac{C}{2D}$ (hence $A = -D$).
    \\The only way for these to be simultaneously true is if $A = D = 0$ which contradicts the initial assumption that $A\neq 0$ and $D\neq 0$.
    \\Therefore a circle that is mapped to a circle never has it's center mapped to the new center \qedsymbol
    \break
    \\Note that if the center of the original circle is the origin then the center remains the origin for the new circle, however under $\frac{1}{z}$ the origin is mapped to $\infty$ so there is still no issue.
\end{center}


\newpage
\section*{100.1}
\begin{center}
    \doublespacing
    Recall that the implicit form of the linear fractional transformation (given below) maps $z_1$ to $w_1$, $z_2$ to $w_2$, and $z_3$ to $w_3$:
    \[\frac{(w - w_1)(w_2 - w_3)}{(w - w_3)(w_2 - w_1)} =\frac{(z - z_1)(z_2 - z_3)}{(z - z_3)(z_2 - z_1)}\]
    Therefore by letting $z_1 = -1$, $z_2 = 0$, $z_3 = 1$, $w_1 = -i$, $w_2 = 1$, and $w_3 = i$ we get the following:
    \[\frac{(w + i)(1 - i)}{(w - i)(1 + i)} =\frac{(z + 1)(0 - 1)}{(z - 1)(0 + 1)}\]
    \[(w + i)(1 - i)(z - 1) = -(z + 1)(1 + i)(w - i)\]
    \[(w - iw + i + 1)(z - 1) = -(z + iz + i + 1)(w - i)\]
    \[wz - iwz + iz + z - w + iw - i - 1 = -zw - izw - iw - w + iz - z - 1 + i\]
    \[wz + z + iw - i = -wz - iw - z + i\]
    \[2(wz + z + iw - i) = 0\]
    \[w(z + i) + (z - i) = 0\]
    \[w =-\frac{z - i}{z + i} =\frac{i - z}{i + z}\]
    \qedsymbol
\end{center}


\newpage
\section*{Problem 2}

{\Large\textbf{a.}} Let $f(z) =\frac{1}{z+1}$ and $\Gamma$ be the contour shown in the problem description.
\begin{center}
    \doublespacing
    Let $D$ be a domain that encompasses $\Gamma$ but that does not contain $-1$, such a domain exists because $-1$ is exterior to $\Gamma$.
    \\Clearly $f$ is analytic in $D$ since $-1\notin D$ and $-1$ is clearly the only point where $f$ is not analytic.
    \\If you imagine a person standing outside of $D$, initially looking to any point $w_0\in\Gamma$ and tracing the path of someone else walking around $\Gamma$, then that person standing outside of $D$ won't rotate any times counterclockwise while tracing the other's path.
    \\Therefore we know the winding number $W(\Gamma,\xi) = 0$ for $\xi\in\mathbb{C}\backslash D$.
    \\Therefore since $0\in D\backslash\Gamma$ and $W(\Gamma,\xi) = 0$ for $\xi\in\mathbb{C}\backslash D$ we know:
    \[\frac{1}{2\pi i}\int _{\Gamma}\frac{f(z)}{z} dz = W(\Gamma, 0) f(0)\]
    Where $W(C, z_0)$ represents the winding number for a contour $C$ and a given point $z_0\notin C$.
    \\If you imagine a person standing at the origin ($z_0 = 0$), initially looking to any point $w_0\in\Gamma$ and tracing the path of someone else walking around $\Gamma$, then that person at the origin will do 3 complete counterclockwise rotations while tracing the other's path.
    \\Therefore we know the winding number $W(\Gamma, 0) = 3$.
    \\So we have that:
    \[\frac{1}{2\pi i}\int _{\Gamma}\frac{f(z)}{z} dz = W(\Gamma, 0) f(0) = 3(\frac{1}{0+1}) = 3\]
    Leaving us with the final result:
    \[\int _{\Gamma}\frac{f(z)}{z} dz =\int _{\Gamma}\frac{1}{z(z+1)} dz = 6\pi i\]
    \qedsymbol
    \break
    \\Part b on next page.
\end{center}

\newpage
{\Large\textbf{b.}} Let $f(z) =\frac{1}{z(z^2 - 1)}$ and $C$ be the contour shown in the problem description.
\begin{center}
    \doublespacing
    What we are going to do is deconstruct this into a number of simple closed contours.
    \\Let $C_1$ be the simple closed contour encompassing only $z_1 = 0$, $\Gamma _1$ be the contour following directly from $C_1$ and ending where $C$ intersects itself next, $C_2$ be the simple closed contour following directly from $\Gamma _1$ and ending where $C$ intersects itself next, $\Gamma _2$ be the contour following directly from $C_2$ and ending where $C$ intersects itself next, and finally let $C_3 =\Gamma _1 +\Gamma _2$.
    \\Note that $C_1$, $C_2$, and $C_3$ are all simple closed contours.
    \\We know first that:
    \[\int _C f(z) dz =\int _{C_1} f(z) dz +\int _{\Gamma _1} f(z) dz +\int _{C_2} f(z) dz +\int _{\Gamma _2} f(z) dz =\int _{C_1} f(z) dz +\int _{C_2} f(z) dz +\int _{C_3} f(z) dz\]
\end{center}
\begin{itemize}
    \item Considering $C_1$:
\end{itemize}
\begin{center}
    \doublespacing
    Let $D_1$ be a domain that encompasses $C_1$ but that does not contain $\pm 1$, such a domain exists because $\pm 1$ are exterior $C_1$.
    \\Clearly $f_1 (z) =\frac{1}{z^2 - 1}$ is analytic in $D_1$ since $\pm 1\notin D_1$ and $\pm 1$ are clearly the only points where $f_1$ is not analytic.
    \\If you imagine a person standing outside of $D_1$, initially looking to any point $w_0\in C_1$ and tracing the path of someone else walking around $C_1$, then that person standing outside of $D_1$ won't rotate any times counterclockwise while tracing the other's path.
    \\Therefore we know the winding number $W(C_1,\xi) = 0$ for $\xi\in\mathbb{C}\backslash D_1$.
    \\Therefore since $0\in D_1\backslash C_1$ and $W(C_1,\xi) = 0$ for $\xi\in\mathbb{C}\backslash D_1$ we know:
    \[\frac{1}{2\pi i}\int _{C_1}\frac{f_1(z)}{z} dz = W(C_1, 0) f_1(0)\]
    Where $W(\Gamma, z_0)$ represents the winding number for a contour $\Gamma$ and a given point $z_0\notin\Gamma$.
    \\If you imagine a person standing at the origin ($z_0 = 0$), initially looking to any point $w_0\in C_1$ and tracing the path of someone else walking around $C_1$, then that person at the origin will do 1 complete clockwise rotation while tracing the other's path.
    \\Therefore we know the winding number $W(C_1, 0) = -1$.
    \\So we have that:
    \[\frac{1}{2\pi i}\int _{C_1}\frac{f_1(z)}{z} dz = W(C_1, 0) f_1(0) = -(\frac{1}{0-1}) = 1\]
    Leaving us with the final result:
    \[\int _{C_1}\frac{f_1(z)}{z} dz =\int _{C_1}\frac{1}{z(z^2-1)} dz = 2\pi i\]
    Continued on next page.
    \newpage
\end{center}
\begin{itemize}
    \item Considering $C_3$:
\end{itemize}
\begin{center}
    \doublespacing
    Let $D_3$ be a domain that encompasses $C_3$ but that does not contain 1 or 0, such a domain exists because 1 and 0 are exterior $C_1$.
    \\Clearly $f_3 (z) =\frac{1}{z(z - 1)}$ is analytic in $D_3$ since $0, 1\notin D_1$ and $0, 1$ are clearly the only points where $f_3$ is not analytic.
    \\If you imagine a person standing outside of $D_3$, initially looking to any point $w_0\in C_3$ and tracing the path of someone else walking around $C_3$, then that person standing outside of $D_3$ won't rotate any times counterclockwise while tracing the other's path.
    \\Therefore we know the winding number $W(C_3,\xi) = 0$ for $\xi\in\mathbb{C}\backslash D_3$.
    \\Therefore since $-1\in D_3\backslash C_3$ and $W(C_3,\xi) = 0$ for $\xi\in\mathbb{C}\backslash D_3$ we know:
    \[\frac{1}{2\pi i}\int _{C_3}\frac{f_3(z)}{z+1} dz = W(C_3, -1) f_3(-1)\]
    Where $W(\Gamma, z_0)$ represents the winding number for a contour $\Gamma$ and a given point $z_0\notin\Gamma$.
    \\If you imagine a person standing at the origin ($z_0 = 0$), initially looking to any point $w_0\in C_3$ and tracing the path of someone else walking around $C_3$, then that person at the origin will do 1 complete counterclockwise rotation while tracing the other's path.
    \\Therefore we know the winding number $W(C_3, -1) = 1$.
    \\So we have that:
    \[\frac{1}{2\pi i}\int _{C_3}\frac{f_3(z)}{z+1} dz = W(C_3, -1) f_3(-1) =\frac{1}{-1(-1 - 1)} =\frac{1}{2}\]
    Leaving us with the final result:
    \[\int _{C_3}\frac{f_3(z)}{z} dz =\int _{C_3}\frac{1}{z(z^2-1)} dz =\pi i\]
    \break
    Continued on next page.
    \newpage
\end{center}
\begin{itemize}
    \item Considering $C_2$:
\end{itemize}
\begin{center}
    \doublespacing
    Clearly $C_2$ is a simple closed contour that encompasses all of the singularities ($0,\pm 1$) of $f(z) =\frac{1}{z(z^2 - 1)}$.
    \\Furthermore we may write $f(z) =\frac{\frac{1}{z^2-1}}{z}$, $f(z) =\frac{\frac{1}{z(z+1)}}{z-1}$, and $f(z) =\frac{\frac{1}{z(z-1)}}{z+1}$.
    \\The numerator of the first representation above is analytic at $z_1 = 0$, similarly that of the second is analytic at $z_2 = 1$, and finally that of the third is analytic at $z_3 = -1$.
    \\Since these are simple poles we may use the fact that $Res_{z = z_0}\frac{\phi (z)}{z - z_0} =\phi (z_0)$ where $\phi (z)$ is analytic at $z_0$.
    \\Therefore:
    \\$Res_{z = 0} f(z) = Res_{z=0}\frac{1}{z(z^2 - 1)} =\frac{1}{z^2 - 1}\Big{|}_{z = 0} = -1$
    \\$Res_{z = -1} f(z) = Res_{z=-1}\frac{1}{z(z^2 - 1)} =\frac{1}{z(z-1)}\Big{|}_{z = -1} =\frac{1}{2}$
    \\$Res_{z = 1} f(z) = Res_{z=1}\frac{1}{z(z^2 - 1)} =\frac{1}{z(z+1)}\Big{|}_{z = 1} =\frac{1}{2}$
    \\Since $C_2$ is a simple closed contour and $f(z) =\frac{1}{z(z^2 - 1)}$ is analytic inside and on $C_2$ except at isolated singularities we know:
    \[\int _{C_2} f(z) dz =\int _{C_2}\frac{1}{z(z^2 - 1)} dz = 2\pi i\Bigg{(}\sum _{k=1}^3 Res_{z = z_k} f(z)\Bigg{)} = 2\pi i\big{(}-1 +\frac{1}{2} +\frac{1}{2}\big{)} = 0\]
    So finally we have that:
    \[\int _{C}\frac{1}{z(z^2 - 1)} dz =\int _{C} f(z) dz =\int _{C_1} f(z) dz +\int _{C_2} f(z) dz +\int _{C_3} f(z) dz = 2\pi i + 0 +\pi i = 3\pi i\]
\end{center}

\end{document}
