\documentclass{article}
\usepackage[utf8]{inputenc}
\usepackage{setspace}
\usepackage[margin=1.5cm]{geometry}
\usepackage{amsmath}
\usepackage{amsthm}
\usepackage{amsfonts}

\title{Irrationality and Sets}
\author{Matthew Seguin}
\date{}

\begin{document}

\maketitle

\section*{1.2.1}

{\Large \textbf{a.}} For the sake of contradiction assume $\sqrt{3}$ is not irrational, that is assume it can be represented as a fraction of integers.
\begin{center}
    \doublespacing
    Then $\sqrt{3} = \frac{p}{q}$ for some p, q $\in \mathbb{Z}$
    \begin{singlespace}
        We can also assume that p and q are relatively prime because if they aren't we simply reduce the fraction until the numerator and denominator are relatively prime. Therefore so far we have that the only common divisor of p and q is 1.
    \end{singlespace}
    Squaring both sides gives $3 = \frac{p^2}{q^2}$
    \\Multiplying both sides by $q^2$ gives $p^2 = 3q^2$
    \\This means that 3 divides $p^2$, and since 3 is not a perfect square this also means that 3 divides p.
    \\So we can write p as $p=3k$ for some k $\in \mathbb{Z}$
    \\Substituting into the previous equation gives $p^2 = (3k)^2 = 9k^2 = 3q^2$
    \\Dividing both sides by 3 gives $3k^2 = q^2$
    \\This means that 3 divides $q^2$, and since 3 is not a perfect square this also means that 3 divides q.
    \begin{singlespace}
        So we have reached a contradiction because we have shown that 3 divides both p and q, hence p and q have the common divisor 3. However, our assumption was that p and q had no common divisor besides 1 since they are relatively prime. Thus our assumption that $\sqrt{3}$ can be represented as a fraction of integers must be false. So $\sqrt{3}$ is irrational. \qedsymbol
        \newline
        \newline A similar argument works for showing that $\sqrt{6}$ is irrational.
        \\You get $6q^2 = p^2$ and hence $p^2$ is divisible by 6 which is not a perfect square so p is also divisible by 6. The rest follows similarly.
    \end{singlespace}
\end{center}

\noindent {\Large \textbf{b.}} You can not use the same process in an attempt to say that $\sqrt{4}$ is irrational.
\begin{center}
    \doublespacing
    If you follow the same process you get $\sqrt{4} = \frac{p}{q}$
    \\Squaring both sides gives $4 = \frac{p^2}{q^2}$
    \\Multiplying both sides by $q^2$ gives $p^2 = 4q^2$
    \\This means 4 divides $p^2$, but that does not imply that 4 divides p. Instead it only implies that 2 divides p.
    \\So we can write p as $p=2k$ for some k $\in \mathbb{Z}$
    \\Substituting into the previous equation gives $p^2 = (2k)^2 = 4k^2 = 4q^2$
    \\Dividing both sides by 4 gives $k^2 = q^2$
    \\There is no contradiction here since k and q can both be 1 and hence p = 2 so $p^2 = 4$
    \\So the same process clearly does not work for attempting to say that $\sqrt{4}$ is irrational.
\end{center}


\newpage
\section*{1.2.5}
\begin{center}
    Let U be an arbitrary universal set.
    \\Recall that for any set S, $S^c = \{x\in U: x\notin S\}$
    \\For any two sets X and Y, $X\cup Y = \{x\in U: x\in X \:or\: x\in Y\}$
    \\And for any two sets X and Y, $X\cap Y = \{x\in X: x\in Y\} = \{y\in Y: y\in X\}$
\end{center}

{\Large \textbf{a.}} Let A and B be subsets of $\mathbb{R}$.
\begin{center}
    \doublespacing
    If $x\in (A\cap B)^c$ then by definition $x\notin A\cap B$.
    \\So if $x\notin A\cap B$ then by definition $x\notin A$ or $x\notin B$.
    \\And since $x\notin A$ or $x\notin B$ by definition $x\in A^c$ or $x\in B^c$.
    \\So by definition $x\in A^c\cup B^c$.
    \\Therefore if an element is in $(A\cap B)^c$ then it is also in $A^c\cup B^c$, so $(A\cap B)^c\subseteq A^c\cup B^c$.
\end{center}

{\Large \textbf{b.}} Let A and B be subsets of $\mathbb{R}$.
\begin{center}
    \doublespacing
    If $x\in A^c\cup B^c$ then by definition $x\in A^c$ or $x\in B^c$.
    \\So if $x\in A^c$ or $x\in B^c$ then by definition $x\notin A$ or $x\notin B$.
    \\And since $x\notin A$ or $x\notin B$ by definition $x\notin A\cap B$.
    \\So by definition $x\in (A\cap B)^c$.
    \\Therefore if an element is in $A^c\cup B^c$ then it is also in $(A\cap B)^c$, so $A^c\cup B^c\subseteq (A\cap B)^c$. \qedsymbol
    \\Since $(A\cap B)^c\subseteq A^c\cup B^c$ and $A^c\cup B^c\subseteq (A\cap B)^c$ every element in $(A\cap B)^c$ is also in $A^c\cup B^c$ and vice versa.
    \\Therefore $(A\cap B)^c = A^c\cup B^c$
\end{center}

{\Large \textbf{c.}} Let A and B be subsets of $\mathbb{R}$.
\begin{center}
    \doublespacing
    \begin{itemize}
        \item If $x\in (A\cup B)^c$ then by definition $x\notin A\cup B$.
        \\So if $x\notin A\cup B$ then by definition $x\notin A$ and $x\notin B$.
        \\And since $x\notin A$ and $x\notin B$ by definition $x\in A^c$ and $x\in B^c$.
        \\So by definition $x\in A^c\cap B^c$.
        \\Therefore if an element is in $(A\cup B)^c$ then it is also in $A^c\cap B^c$, so $(A\cup B)^c\subseteq A^c\cap B^c$.
        \item If $x\in A^c\cap B^c$ then by definition $x\in A^c$ and $x\in B^c$.
        \\So if $x\in A^c$ and $x\in B^c$ then by definition $x\notin A$ and $x\notin B$.
        \\And since $x\notin A$ and $x\notin B$ by definition $x\notin A\cup B$.
        \\So by definition $x\in (A\cup B)^c$.
        \\Therefore if an element is in $A^c\cap B^c$ then it is also in $(A\cup B)^c$, so $A^c\cap B^c\subseteq (A\cup B)^c$.
    \end{itemize}
    Since $(A\cup B)^c\subseteq A^c\cap B^c$ and $A^c\cap B^c\subseteq (A\cup B)^c$ every element in $(A\cup B)^c$ is also in $A^c\cap B^c$ and vice versa.
    \\Therefore $(A\cup B)^c = A^c\cap B^c$ \qedsymbol
\end{center}


\newpage
\section*{1.2.7}
\begin{center}
    Let $f$ be a function $f: S \rightarrow T$, and $A\subseteq S.$
    \\Define $f(A) = \{f(x)\in T: x\in A\}$
    \\Recall that a real interval $[a, b] = \{x\in \mathbb{R}: a \leq x \leq b\}$
\end{center}

{\Large \textbf{a.}} Let $f(x) = x^2$, A = [0, 2], and B = [1, 4].
\begin{center}
    \doublespacing
    \begin{singlespace}
        Since $f(x)$ is strictly increasing on $[0, \infty)$, the endpoints of the intervals A and B will provide the minimum and maximum values of $f(A)$ and $f(B)$.
    \end{singlespace}
    \begin{itemize}
        \item Therefore $f(A) = [0, 4]$ and $f(B) = [1, 16]$.
        \item \begin{singlespace}
            Furthermore $A\cap B = [1, 2]$, so $f(A\cap B) = [1, 4]$.
            \\And $f(A)\cap f(B) = [1, 4]$, so $f(A\cap B) = f(A)\cap f(B)$ here.
            \end{singlespace}
        \item \begin{singlespace}
            Furthermore $A\cup B = [0, 4]$, so $f(A\cup B) = [0, 16]$.
        \\And $f(A)\cup f(B) = [0, 16]$, so $f(A\cup B) = f(A)\cup f(B)$ here.
        \end{singlespace}
    \end{itemize}
\end{center}

{\Large \textbf{b.}} Let $f(x)$ be as before, $A = [-1, 0]$, and $B = [0, 1]$.
\begin{center}
    \doublespacing
    \begin{singlespace}
        As before $f(x)$ is strictly increasing on $[0, \infty)$, but since I am introducing a negative domain I will point out 
        \\that $f(x)$ is strictly decreasing on $(-\infty, 0)$. So as before the endpoints of the intervals A and B will provide 
        \\the minimum and maximum values of $f(A)$ and $f(B)$.
    \end{singlespace}
    So $f(A) = [0, 1] = f(B)$ and therefore $f(A)\cap f(B) = [0, 1]$.
    \\Furthermore $A\cap B = \{0\}$ so $f(A\cap B) = \{f(0)\} = \{0\}$.
    \\Clearly $1\in f(A)\cap f(B)$ but $1\notin f(A\cap B)$, so $f(A\cap B)\neq f(A)\cap f(B)$ here.
\end{center}

{\Large \textbf{c.}} Let $g$ be an arbitrary function $g: \mathbb{R} \rightarrow \mathbb{R}$ and $A$ and $B$ be arbitrary subsets of $\mathbb{R}$.
\begin{center}
    \doublespacing
    Let $x\in g(A\cap B)$. Then $g(y) = x$ for some $y\in A\cap B$.
    \\For this $y$ since $y\in A\cap B$, $y\in A$ and $y\in B$.
    \\So $g(y) = x\in g(A)$ and $g(y) = x\in g(B)$, therefore $g(y) = x\in g(A)\cap g(B)$.
    \\Therefore if $x\in g(A\cap B)$ then $x\in g(A)\cap g(B)$, so $g(A\cap B)\subseteq g(A)\cap g(B)$ for an arbitrary function $g: \mathbb{R} \rightarrow \mathbb{R}$ \qedsymbol
\end{center}

{\Large \textbf{d.}} Let $g$ be an arbitrary function $g: \mathbb{R} \rightarrow \mathbb{R}$ and $A$ and $B$ be arbitrary subsets of $\mathbb{R}$.
\begin{center}
    \doublespacing
    I claim that $g(A\cup B) = g(A)\cup g(B)$ for an arbitrary function $g: \mathbb{R} \rightarrow \mathbb{R}$ and arbitrary subsets $A$ and $B$ of $\mathbb{R}$.
    \begin{itemize}
        \item Let $x\in g(A\cup B)$. Then $g(y) = x$ for some $y\in A\cup B$.
        \\For this $y$ since $y\in A\cup B$, $y\in A$ or $y\in B$.
        \\So $g(y) = x\in g(A)$ or $g(y) = x\in g(B)$, therefore $g(y) = x\in g(A)\cup g(B)$.
        \\Therefore if $x\in g(A\cup B)$ then $x\in g(A)\cup g(B)$, so $g(A\cup B)\subseteq g(A)\cup g(B)$.
        \item Let $x\in g(A)\cup g(B)$. Then $x\in g(A)$ or $x\in g(B)$.
        \\So $g(y) = x$ for some $y\in A$ or $g(z) = x$ for some $z\in B$.
        \\So $g(w) = x$ for some $w\in A\cup B$, therefore $x\in g(A\cup B)$.
        \\Therefore if $x\in g(A)\cup g(B)$ then $x\in g(A\cup B)$, so $g(A)\cup g(B)\subseteq g(A\cup B)$.
    \end{itemize}
    Therefore since $g(A\cup B)\subseteq g(A)\cup g(B)$ and $g(A)\cup g(B)\subseteq g(A\cup B)$, $g(A\cup B) = g(A)\cup g(B)$ for an arbitrary \\function $g: \mathbb{R} \rightarrow \mathbb{R}$ \qedsymbol
\end{center}


\newpage
\section*{1.2.12}
\begin{center}
    Recall that a relation $>$ on a set $S$ is transitive if given $a, b, c\in S$ where $a > b$ and $b > c$ then $a > c$.
    \\Let $y_1 = 6$ and for each $n\in \mathbb{N}$ define $y_{n+1} = \frac{(2y_n - 6)}{3}$.
\end{center}

{\Large \textbf{a.}} Let $S = \{n\in \mathbb{N}: y_n > -6\}$.
\begin{center}
    \doublespacing
    \begin{itemize}
        \item Base Case: 
        \\We know $y_1 = 6 > -6$, so $1\in S$.
        \item Inductive Step:
        \\Assume that $n\in S$, then $y_n > -6$
        \\Multiplying both sides by 2 we get $2y_n > -12$
        \\Subtracting 6 from both sides we get $2y_n - 6 > -18$
        \\Dividing both sides by 3 we get $\frac{2y_n - 6}{3} > -6$
        \\Since $y_{n+1} = \frac{(2y_n - 6)}{3}$, we have that $y_{n+1} > -6$ and therefore $n+1\in S$.
        \\So for any $n\in \mathbb{N}$ if $n\in S$ then $n+1\in S$.
    \end{itemize}
    Therefore since $1\in S$ and if $n\in S$ then $n+1\in S$, $S = \mathbb{N}$. So $y_n > -6$ for all $n\in \mathbb{N}$ \qedsymbol
\end{center}

{\Large \textbf{b.}} Let $S = \{n\in \mathbb{N}: y_n > y_{n+1}\}$.
\begin{center}
    \doublespacing
    \begin{itemize}
        \item Base Case:
        \\We know $y_1 = 6$, so $y_2 = \frac{(2y_1 - 6)}{3} = 2$
        \\Therefore since $y_1 = 6 > 2 = y_2$, $1\in S$.
        \item Inductive Step:
        \\Assume that $n\in S$, then $y_n > y_{n+1}$
        \\Multiplying both sides by 2 we get $2y_n > 2y_{n+1}$
        \\Subtracting 6 from both sides we get $2y_n - 6 > 2y_{n+1} - 6$
        \\Dividing both sides by 3 we get $\frac{2y_n - 6}{3} > \frac{2y_{n+1} - 6}{3}$
        \\Since $y_{n+1} = \frac{(2y_n - 6)}{3}$ and $y_{n+2} = \frac{(2y_{n+1} - 6)}{3}$, we have that $y_{n+1} > y_{n+2}$ and therefore $n+1\in S$.
        \\So for any $n\in \mathbb{N}$ if $n\in S$ then $n+1\in S$.
    \end{itemize}
    Therefore since $1\in S$ and if $n\in S$ then $n+1\in S$, $S = \mathbb{N}$. So $y_n > y_{n+1}$ for all $n\in \mathbb{N}$.
    \\Since $>$ is a transitive relation on $\mathbb{R}$ we have that if $n, m\in \mathbb{N}$ and $n > m$ then $y_n < y_m$. 
    \\So $y_1 > y_2 > y_3 > ...$, and therefore the sequence $(y_1, y_2, y_3, ...)$ is decreasing \qedsymbol
\end{center}


\newpage
\section*{1.2.13}
\begin{center}
    Recall from problem 1.2.5 that $(A\cup B)^c = A^c\cap B^c$.
\end{center}

{\Large \textbf{a.}} Let $S = \{n\in \mathbb{N}: (A_1\cup A_2\cup ...\cup A_n)^c = A^c_1\cap A^c_2\cap ...\cap A^c_n \;for \;arbitrary \;sets \;A_1, A_2, ... A_n\}$
\begin{center}
    \doublespacing
    \begin{itemize}
        \item Base Case:
        \\We know for a set $A_1$ that $(A_1)^c = A^c_1$, so $1\in \mathbb{N}$.
        \item Inductive Step:
        \\Assume that $n\in S$, then $(A_1\cup A_2\cup ...\cup A_n)^c = A^c_1\cap A^c_2\cap ...\cap A^c_n$.
        \\Consider another arbitrary set $A_{n+1}$. Let $B = A_1\cup A_2\cup ...\cup A_n$.
        \\Then $(B\cup A_{n+1})^c = B^c\cap A^c_{n+1}$ by the result of problem 1.2.5.
        \\Since $B = A_1\cup A_2\cup ...\cup A_n$ and $B^c = (A_1\cup A_2\cup ...\cup A_n)^c = A^c_1\cap A^c_2\cap ...\cap A^c_n$, we have 
        \\that $(A_1\cup A_2\cup ...\cup A_n\cup A_{n+1})^c = (B\cup A_{n+1})^c = B^c\cap A^c_{n+1} = A^c_1\cap A^c_2\cap ...\cap A^c_n\cap A^c_{n+1}$. Therefore $n+1\in S$.
        \\So for any $n\in \mathbb{N}$ if $n\in S$ then $n+1\in S$.
    \end{itemize}
    Therefore since $1\in S$ and if $n\in S$ then $n+1\in S$, $S = \mathbb{N}$. So $(A_1\cup A_2\cup ...\cup A_n)^c = A^c_1\cap A^c_2\cap ...\cap A^c_n$ for all $n\in \mathbb{N}$ \qedsymbol
\end{center}

{\Large \textbf{b.}} Let $B_1 = \{\frac{1}{x}\in \mathbb{Q}: x\in \mathbb{N},\: x\geq 1\}$, $B_2 = \{\frac{1}{x}\in \mathbb{Q}: x\in \mathbb{N},\: x\geq 2\}$, ... , $B_n = \{\frac{1}{x}\in \mathbb{Q}: x\in \mathbb{N},\: x\geq n\}$, ...
\begin{center}
    \doublespacing
    Let $S = \{n\in \mathbb{N}: B_1\cap B_2\cap ...\cap B_n \neq \phi \}$
    \begin{itemize}
        \item Finite Case:
        \begin{itemize}
            \item Base Case:
            \\$B_1 = \{1, \frac{1}{2}, \frac{1}{3}, ...\}$ so $1\in B_1$ but $1\notin \phi$, so $B_1 \neq \phi$. Therefore $1\in S$.
            \item Inductive Step:
            \\Note that $B_1\supset B_2\supset B_3\supset ...$, so $B_1\cap B_2\cap B_3\cap ...\cap B_n = B_n$
            \\Assume that $n\in S$, then $B_1\cap B_2\cap B_3\cap ...\cap B_n \neq \phi$.
            \\Then consider $B_{n+1}$, since $B_1\cap B_2\cap B_3\cap ...\cap B_n \neq \phi$ we know $B_1\cap B_2\cap B_3\cap ...\cap B_n = B_n$.
            \\And since $B_{n+1}\subset B_n$, $B_{n+1}\cap B_n = B_{n+1}$. So $\frac{1}{n+1}\in B_1\cap B_2\cap B_3\cap ...\cap B_n\cap B_{n+1}$ while $\frac{1}{n+1}\notin \phi$.
            \\So $B_1\cap B_2\cap B_3\cap ...\cap B_n\cap B_{n+1}\neq \phi$, and therefore $n+1\in S$.
            \\So for any $n\in \mathbb{N}$ if $n\in S$ then $n+1\in S$.
        \end{itemize}
        Therefore since $1\in S$ and if $n\in S$ then $n+1\in S$, $S = \mathbb{N}$. So $B_1\cap B_2\cap B_3\cap ...\cap B_n \neq \phi$ for all $n\in \mathbb{N}$. \qedsymbol
        \item Infinite Case:
        \\Let $A = \cap _{j=1}^{\infty} B_j = B_1\cap B_2\cap B_3\cap ...$, and let $n\in \mathbb{N}$.
        \\Say for the sake of contradiction that $x = \frac{1}{n}\in A$, then $x\in B_1\cap B_2\cap B_3\cap ...$ so $x\in B_1 \;and\; x\in B_2 \;and\; x\in B_3 \;and\; ...$
        \\But consider $B_{n+1} = \{\frac{1}{n+1}, \frac{1}{n+2}, \frac{1}{n+3}, ...\}$, clearly $x = \frac{1}{n}\notin B_{n+1}$. So we have a contradiction.
        \\Therefore our assumption that $x = \frac{1}{n}\in A$ is false, and there are no elements in $A$ since any element in $A$ would be of the form $\frac{1}{n}$ for some $n\in \mathbb{N}$. So $A = \cap _{j=1}^{\infty} B_j = \phi$ \qedsymbol
    \end{itemize}
\end{center}

\newpage
{\Large \textbf{c.}} Let $A_1, A_2, A_3, ...$ be arbitrary sets.
\begin{center}
    \doublespacing
    Consider $(\cup _{j=1}^{\infty} A_j)^c$ and $\cap _{j=1}^{\infty} A^c_j$
    \begin{itemize}
        \item If $x\in (\cup _{j=1}^{\infty} A_j)^c$, then $x\notin \cup _{j=1}^{\infty} A_j$, so $x\notin A_1 \;and\; x\notin A_2 \;and\; x\notin A_3 \;and\; ...$
        \\Therefore $x\in A^c_1 \;and\; x\in A^c_2 \;and\; x\in A^c_3 \;and\; ...$, so $x\in A^c_1\cap A^c_2\cap A^c_3\cap ... = \cap _{j=1}^{\infty} A^c_j$
        \\So $(\cup _{j=1}^{\infty} A_j)^c\subseteq \cap _{j=1}^{\infty} A^c_j$
        \item If $x\in \cap _{j=1}^{\infty} A_j^c$, then $x\in A^c_1 \;and\; x\in A^c_2 \;and\; x\in A^c_3 \;and\; ...$, so $x\notin A_1 \;and\; x\notin A_2 \;and\; x\notin A_3 \;and\; ...$
        \\Therefore $x\notin A_1\cup A_2\cup A_3\cup ...$, so $x\in (A_1\cup A_2\cup A_3\cup ...)^c = (\cup _{j=1}^{\infty} A_j)^c$
        \\So $\cap _{j=1}^{\infty} A_j^c\subseteq (\cup _{j=1}^{\infty} A_j)^c$
    \end{itemize}
    Therefore every element in $(\cup _{j=1}^{\infty} A_j)^c$ is in $\cap _{j=1}^{\infty} A^c_j$ and vice versa.
    \\So $(\cup _{j=1}^{\infty} A_j)^c = \cap _{j=1}^{\infty} A^c_j$ \qedsymbol
\end{center}

\end{document}
