\documentclass{article}
\usepackage{graphicx} % Required for inserting images
\usepackage[utf8]{inputenc}
\usepackage{setspace}
\usepackage[margin=1.5cm]{geometry}
\usepackage{amsmath}
\usepackage{amsthm}
\usepackage{amsfonts}
\usepackage{indentfirst}

\title{Sequences and Convergence}
\author{Matthew Seguin}
\date{}

\begin{document}

\maketitle


\section*{2.2.8} 
\begin{center}
    $(x_n)$ is "zero-heavy" if there exists an $M\in\mathbb{N}$ such that for all $N\in\mathbb{N}$ there exists an $n\in\mathbb{N}$ where $N\leq n\leq N + M$ such that $x_n = 0$.
\end{center}

{\Large \textbf{a.}} Let $(a_n) = (0, 1, 0, 1, 0, 1, 0, 1, ...)$ where for $n\in\mathbb{N}$, $a_n = 0$ if $n$ is odd and $a_n = 1$ if $n$ is even.
\begin{center}
    \doublespacing
    By our definition of zero-heavy this sequence $a_n$ is zero heavy.
    \\Let $M = 1$ then since for all $n\in\mathbb{N}$ either $a_n = 0$ or $a_{n+1} = 0$ so we have found an $M$ such that for all $N\in\mathbb{N}$ there exists an $n$ where $N\leq n\leq N + M$ such that $a_n = 0$ so $(a_n) = (0, 1, 0, 1, 0, 1, ...)$ is zero-heavy \qedsymbol
\end{center}

{\Large \textbf{b.}} If a sequence is zero-heavy it must contain an infinite number of zeros.
\begin{center}
    \doublespacing
    Proof:
    \\Let $(a_n)$ be a sequence that contains only a finite number of zeros.
    \\Then for some $m\in\mathbb{N}$ we know that $a_m$ is the final zero in the sequence. Consider $N\in\mathbb{N}$ where $N > m$ then $a_N\neq 0$.
    \\So for any $M\in\mathbb{N}$ there does not exist an $n\in\mathbb{N}$ where $N\leq n\leq N + M$ and $a_n = 0$ since $a_m$ is the final zero and $n > m$.
    \\Consequently this $(a_n)$ can not be zero-heavy.
    \\This was for an arbitrary sequence with finitely many zeros so it is true for all such sequences.
    \\So if a sequence $(x_n)$ contains only a finite number of zeros it cannot be zero-heavy. Therefore if a sequence is zero-heavy it must contain an infinite number of zeros \qedsymbol
\end{center}

{\Large \textbf{c.}} Let $(a_n)$ be a sequence such that the distance between zero number $n$ and zero number $n + 1$ is strictly increasing.
\begin{center}
    \doublespacing
    Then for any $M\in\mathbb{N}$ we can choose $N\in\mathbb{N}$ large enough so there are at least $M + 1$ non-zero terms until the next zero.
    \\So there does not exist an $M\in\mathbb{N}$ such that for all $N\in\mathbb{N}$ there exists an $n\in\mathbb{N}$ where $N\leq n\leq N + M$ such that $x_n = 0$ and therefore $(a_n)$ is not zero heavy yet contains an infinite number of zeros.
    \\An example of such a sequence is $(a_n) = (0, 1, 0, 1, 2, 0, 1, 2, 3, 0, 1, 2, 3, 4, 0, ...)$.
    \\Clearly there are $n$ terms between zero $n$ and $n+1$ for all $n\in\mathbb{N}$. So for any choice of $M\in\mathbb{N}$ we can find an $N\in\mathbb{N}$ such that $a_N$ has at least $M + 1$ non-zero terms before the next zero. Therefore $(a_n)$ is not zero-heavy.
\end{center}

{\Large \textbf{d.}} The logical negation of the definition of zero-heavy is as follows:
\begin{center}
    \doublespacing
    A sequence $(x_n)$ is not zero-heavy if for all $M\in\mathbb{N}$ there exists an $N\in\mathbb{N}$ where for all $n\in\mathbb{N}$ where $N\leq n\leq N + M$ it is the case that $x_n\neq 0$.
\end{center}


\newpage
\section*{2.3.1}

{\Large \textbf{b.}} Let $x_n\geq 0$ for all $n\in\mathbb{N}$ and $(x_n)\rightarrow x$. Then the sequence $(\sqrt{x_n})$ exists.
\begin{center}
    \doublespacing
    \begin{itemize}
        \item First I would like to prove 2.3.1 part a. since it removes the issues of dealing with 0.
        \\Say $(x_n)\rightarrow 0$ then let $\epsilon > 0$ and $\alpha = \epsilon ^2$. Then $\alpha > 0$.
        \\So there exists an $N\in\mathbb{N}$ such that when $n\geq N$ we have $|x_n - 0| = x_n <\alpha$.
        \\Now we have if $n\geq N$ then $\sqrt{x_n} <\sqrt{\alpha} =\epsilon$.
        \\Therefore there exists an $N\in\mathbb{N}$ such that when $n\geq N$ we have $|\sqrt{x_n} - 0| <\epsilon$.
        \\This was done for arbitrary $\epsilon > 0$ and is therefore true for all $\epsilon > 0$.
        \\So we have shown that for all $\epsilon > 0$ there exists an $N\in\mathbb{N}$ such that if $n\geq N$ then $|\sqrt{x_n} - 0| <\epsilon$.
        \\Therefore if $x_n\geq 0$ for all $n\in\mathbb{N}$ and $(x_n)\rightarrow 0$ then $(\sqrt{x_n})\rightarrow 0$.
        \item Now let $(x_n)\rightarrow x\neq 0$. It must be that $x > 0$ by the Order Limit Theorem and our assumption that $x\neq 0$.
        \\We already know for all $\epsilon > 0$ there exists an $N\in\mathbb{N}$ such that when $n\geq N$ we have $|x_n - x| <\epsilon$.
        \\Furthermore $x_n - x = (\sqrt{x_n} -\sqrt{x})(\sqrt{x_n} +\sqrt{x})$ for all $n\in\mathbb{N}$.
        \\We want to get that for all $\epsilon > 0$ there exists an $N\in\mathbb{N}$ such that when $n\geq N$ we have $|\sqrt{x_n} -\sqrt{x}| <\epsilon$.
        \\Let $\epsilon > 0$ then consider $\alpha = (\sqrt{x_n} +\sqrt{x})\epsilon$. Since $\sqrt{x_m} +\sqrt{x} > 0$ for all $m\in\mathbb{N}$ we have that $\alpha = (\sqrt{x_n} +\sqrt{x})\epsilon > 0$.
        \\Then since $(x_n)\rightarrow x$ we know there exists an $N\in\mathbb{N}$ such that when $n\geq N$ we have $|x_n - x| <\alpha = (\sqrt{x_n} +\sqrt{x})\epsilon$.
        \\So we have that there exists an $N\in\mathbb{N}$ such that if $n\geq N$ then $|x_n - x| = |(\sqrt{x_n} -\sqrt{x})(\sqrt{x_n} +\sqrt{x})| < (\sqrt{x_n} +\sqrt{x})\epsilon$.
        \\Therefore if $n\geq N$ then $|(\sqrt{x_n} -\sqrt{x})(\sqrt{x_n} +\sqrt{x})| = |\sqrt{x_n} -\sqrt{x}||\sqrt{x_n} +\sqrt{x}| < (\sqrt{x_n} +\sqrt{x})\epsilon$.
        \\Since $\sqrt{x_n} +\sqrt{x} > 0$ we have that $|\sqrt{x_n} +\sqrt{x}| = \sqrt{x_n} +\sqrt{x}$ and that we can divide by $\sqrt{x_n} +\sqrt{x}$.
        \\So if $n\geq N$ then $|\sqrt{x_n} -\sqrt{x}||\sqrt{x_n} +\sqrt{x}| = (\sqrt{x_n} +\sqrt{x})|\sqrt{x_n} -\sqrt{x}| < (\sqrt{x_n} +\sqrt{x})\epsilon$.
        \\Dividing by $\sqrt{x_n} +\sqrt{x}$ we have that there exists an $N\in\mathbb{N}$ such that if $n\geq N$ then $|\sqrt{x_n} -\sqrt{x}| <\epsilon$.
        \\This was done for an arbitrary $\epsilon > 0$ and is therefore true for all $\epsilon > 0$.
        \\So we have shown that for all $\epsilon > 0$ there exists an $N\in\mathbb{N}$ such that if $n\geq N$ then $|\sqrt{x_n} -\sqrt{x}| <\epsilon$.
        \\Therefore if $x_n\geq 0$ for all $n\in\mathbb{N}$ and $(x_n)\rightarrow x\neq 0$ then $(\sqrt{x_n})\rightarrow\sqrt{x}$.
    \end{itemize}
    So we have shown that if $x_n\geq 0$ for all $n\in\mathbb{N}$ and $(x_n)\rightarrow 0$ then $(\sqrt{x_n})\rightarrow 0$.
    \\We have also shown that if $x_n\geq 0$ for all $n\in\mathbb{N}$ and $(x_n)\rightarrow x\neq 0$ then $(\sqrt{x_n})\rightarrow\sqrt{x}$.
    \\Since $\sqrt{0} = 0$ we have therefore shown that if $x_n\geq 0$ for all $n\in\mathbb{N}$ and $(x_n)\rightarrow x$ then $(\sqrt{x_n})\rightarrow\sqrt{x}$ \qedsymbol
\end{center}


\newpage
\section*{2.3.3}
\begin{center}
    \doublespacing
    Let $(x_n), (y_n), (z_n)$ be sequences such that $(x_n)\rightarrow l, (z_n)\rightarrow l$, and $x_n\leq y_n\leq z_n$ for all $n\in\mathbb{N}$.
    \\We want to show for all $\epsilon > 0$ there exists an $N\in\mathbb{N}$ such that when $n\geq N$ then $|y_n - l| <\epsilon$.
    \\Recall the triangle inequality $|a+b|\leq |a| + |b|$. Consider $|y_n - l|$.
    \\Then  $|y_n - l| = |(y_n - x_n) + (x_n - l)|\leq |y_n - x_n| + |x_n - l|$.
    \\Then since $x_n\leq y_n\leq z_n$ for all $n\in\mathbb{N}$ we have $0\leq y_n - x_n\leq z_n - x_n$ for all $n\in\mathbb{N}$.
    \\Since $0\leq y_n - x_n\leq z_n - x_n$ we have that $0\leq |y_n - x_n|\leq |z_n - x_n|$.
    \\So we have $|y_n - l|\leq |y_n - x_n| + |x_n - l|\leq |z_n - x_n| + |x_n - l| = |(z_n - l) + (l - x_n)| + |x_n - l|$.
    \\Again applying the triangle inequality we have $|y_n - l|\leq |z_n - l| + |l - x_n| + |x_n - l| = |z_n - l| + 2|x_n - l|$.
    \\Let $\epsilon > 0$ and $\alpha = \epsilon /3$ then $\alpha > 0$. Since $(x_n)\rightarrow l$ and $(z_n)\rightarrow l$ we can find an $N\in\mathbb{N}$ such that when $n\geq N$ then $|x_n - l| <\alpha$ and $|z_n - l| <\alpha$ simultaneously.
    \\So we can find an $N\in\mathbb{N}$ such that when $n\geq N$ then $|y_n - l|\leq |z_n - l| + 2|x_n - l| < 3\alpha =\epsilon$.
    \\This was for arbitrary $\epsilon > 0$ and is therefore true for all $\epsilon > 0$.
    \\So we have shown that for all $\epsilon > 0$ there exists an $N\in\mathbb{N}$ such that if $n\geq N$ then $|y_n - l| <\epsilon$.
    \\Therefore if $(x_n)\rightarrow l, (z_n)\rightarrow l$, and $x_n\leq y_n\leq z_n$ for all $n\in\mathbb{N}$ then $(y_n)\rightarrow l$ \qedsymbol
\end{center}


\newpage
\section*{2.3.7}

{\Large \textbf{a.}} Let $(x_n) = ((-1)^n) = (-1, 1, -1, 1, -1, 1, ...)$ and $(y_n) = ((-1)^{n+1}) = (1, -1, 1, -1, 1, -1, ...)$.
\begin{center}
    \doublespacing
    Both $(x_n)$ and $(y_n)$ diverge as shown many times in previous samples.
    \\However, $(z_n) = (x_n + y_n) = ((-1)^n + (-1)^{n+1}) = (-1 + 1, 1 - 1, -1 + 1, 1 - 1, ...) = (0, 0, 0, 0, ...)$ clearly converges.
    \\Proof: Let $\epsilon > 0$ then let $N = 1$ now if $n\geq N$ then $|z_n - 0| = |0 - 0| = 0 <\epsilon$ so $(z_n)\rightarrow 0$.
    \\So this is such an example of two divergent sequences whose sum converges.
\end{center}

{\Large \textbf{b.}} This is not possible. Let $(x_n)\rightarrow x$ for some $x\in\mathbb{R}$ and $(y_n)$ diverge.
\begin{center}
    \doublespacing
    By the algebraic limit theorem. $(-x_n)\rightarrow -x$.
    \\Assume for the sake of contradiction that $(x_n + y_n)\rightarrow z$ for some $z\in\mathbb{R}$.
    \\Then by the algebraic limit theorem $(x_n + y_n - x_n) = (y_n)\rightarrow z - x$.
    \\But $z - x\in\mathbb{R}$ therefore since $(y_n)$ diverges it can not be that $(y_n)\rightarrow z - x$ so we have a contradiction.
    \\Therefore if $(x_n)$ converges and $(y_n)$ diverges it can not be that $(x_n + y_n)$ converges \qedsymbol
\end{center}

{\Large \textbf{c.}} Let $(b_n) = (\frac{1}{n}) = (1,\frac{1}{2},\frac{1}{3},\frac{1}{4}, ...)$. Then $b_n\neq 0$ for all $n\in\mathbb{N}$. Furthermore $(b_n)\rightarrow 0$.
\begin{center}
    \doublespacing
    \begin{itemize}
        \item Proving $(b_n)\rightarrow 0$:
        \\Let $\epsilon > 0$ then by the Archimedean property there exists an $N\in\mathbb{N}$ such that $\frac{1}{N} <\epsilon$.
        \\Then $|b_N - 0| = b_N =\frac{1}{N} <\epsilon$. Since $(b_n)$ is strictly decreasing if $b_N <\epsilon$ then $b_n <\epsilon$ for $n\geq N$.
        \\Therefore for all $\epsilon > 0$ there exists an $N\in\mathbb{N}$ such that if $n\geq N$ then $|b_n - 0| <\epsilon$. So $(b_n)\rightarrow 0$.
        \item Showing $(\frac{1}{b_n})$ diverges:
        \\The sequence $(\frac{1}{b_n}) = (\frac{1}{1/n}) = (n)$ is unbounded since $\mathbb{N}$ is unbounded.
        \\Since every convergent sequence is bounded it can not be that $(\frac{1}{b_n}) = (n)$ converges.
    \end{itemize}
    So this is such an example of a sequence $(b_n)$ where $b_n\neq 0$ for all $n\in\mathbb{N}$ such that $(\frac{1}{b_n})$ diverges.
\end{center}

{\Large \textbf{d.}} This is not possible. Let $(a_n)$ be unbounded and $(b_n)\rightarrow b$.
\begin{center}
    \doublespacing
    Since $(a_n)$ is unbounded there does not exist an $M\in\mathbb{N}$ such that $|a_n|\leq M$ for all $n\in\mathbb{N}$.
    \\Since $(b_n)$ converges it is bounded so there exists an $N\in\mathbb{N}$ such that $|b_n|\leq N$ for all $n\in\mathbb{N}$.
    \\Therefore there does not exist an $M\in\mathbb{N}$ such that $|a_n| + N\leq M + N$ for all $n\in\mathbb{N}$.
    \\We know $|b_n|\leq N$ for all $n\in\mathbb{N}$ and from the triangle inequality that $|a_n - b_n|\leq |a_n| + |b_n|$.
    \\So there does not exist an $M\in\mathbb{N}$ such that $|a_n - b_n|\leq |a_n| + |b_n|\leq |a_n| + N\leq M + N$ for all $n\in\mathbb{N}$.
    \\Let $L = M + N$. Then we have that there does not exist an $L\in\mathbb{N}$ such that $|a_n - b_n|\leq L$ for all $n\in\mathbb{N}$.
    \\So if $(a_n)$ is unbounded and $(b_n)$ converges then $(a_n - b_n)$ is unbounded \qedsymbol
\end{center}

{\Large \textbf{e.}} Let $(a_n) = (\frac{1}{n}) = (1,\frac{1}{2},\frac{1}{3},\frac{1}{4}, ...)$ and $(b_n) = (n) = (1, 2, 3, 4, ...)$.
\begin{center}
    \doublespacing
    As shown in 2.3.7.c the sequence $(a_n)\rightarrow 0$. Now consider $(a_n b_n) = (\frac{1}{n} n) = (1)$.
    \\Let $\epsilon > 0$ then let $N = 1$ now if $n\geq N$ then $|a_n b_n - 1| = |1 - 1| = 0 <\epsilon$ so $(a_n b_n)\rightarrow 1$.
    \\$(b_n) = (n)$ is unbounded since $\mathbb{N}$ is unbounded. Since all convergent sequences are bounded, $(b_n)$ can not converge.
    \\Therefore this is such an example of sequences $(a_n)$ and $(b_n)$ such that $(a_n)$ and $(a_n b_n)$ converge but $(b_n)$ diverges.
\end{center}


\newpage
\section*{2.3.11}

{\Large \textbf{a.}} Let $(x_n)\rightarrow x$ and $(y_n) = (\frac{\sum _{i=1}^n x_n}{n}) = (x_1,\frac{x_1 + x_2}{2},\frac{x_1 + x_2 + x_3}{3}, ...)$.
\begin{center}
    \doublespacing
    We want to show for all $\epsilon > 0$ there exists an $N\in\mathbb{N}$ such that if $n\geq N$ then $|y_n - x| <\epsilon$.
    \\Since $(x_n)$ converges it is bounded so there exists an $L\in\mathbb{N}$ such that $|x_n|\leq L$ for all $n\in\mathbb{N}$.
    \\So $|x_n| + |x|\leq L + |x|$, and $|x_n - x|\leq |x_n| + |x|\leq L + |x|$ for all $n\in\mathbb{N}$.
    \\Let $\beta > L + |x|$ then we have that $|x_n - x|\leq L + |x| <\beta$ for all $n\in\mathbb{N}$.
    \\Let $\epsilon > 0$ and $\alpha =\epsilon / 2$. Then we know there exists an $N\in\mathbb{N}$ such that for $n\geq N$, $|x_n - x| <\alpha$.
    \\Let $K = N - 1$ then we have for $n > K$, $|x_n - x| <\alpha$.
    \\So $|y_n - x| = |\frac{x_1 + ... + x_n}{n} -\frac{nx}{n}| = |\frac{(x_1 - x) + ... + (x_K - x) + ...+ (x_n - x)}{n}|\leq |\frac{(x_1 - x) + ... + (x_K - x)}{n}| + |\frac{(x_{K+1} - x) + ... + (x_n - x)}{n}|$
    \\And $|y_n - x|\leq |\frac{(x_1 - x) + ... + (x_K - x)}{n}| + |\frac{(x_{K+1} - x) + ... + (x_n - x)}{n}|\leq |\frac{|x_1 - x| + ... + |x_K - x|}{n}| + |\frac{|x_{K+1} - x| + ... + |x_n - x|}{n}|$.
    \\So for $n > K$: $|y_n - x|\leq |\frac{|x_1 - x| + ... + |x_K - x|}{n}| + |\frac{|x_{K+1} - x| + ... + |x_n - x|}{n}| < |\frac{K\beta}{n}| + |\frac{(n-K)\alpha}{n}| < \frac{K\beta}{n} + \alpha$.
    \\Since $K$ and $\beta$ are fixed we can choose an $M$ large enough such that $\frac{K\beta}{n} <\alpha$ for $n > M$.
    \\So when both $n > N$ and $n > M$: $|y_n - x| < \frac{K\beta}{n} +\alpha < 2\alpha =\epsilon$.
    \\Let $J = 1 + max\{N, M\}$. Then for $n\geq J$ we have $|y_n - x| <\epsilon$.
    \\This was for arbitrary $\epsilon > 0$ and is therefore true for all $\epsilon > 0$.
    \\So we have shown that for all $\epsilon > 0$ there exists a $J\in\mathbb{N}$ such that $|y_n - x| <\epsilon$ for $n\geq J$.
    \\Therefore if $(x_n)\rightarrow x$ then $(y_n) = (\frac{\sum _{i=1}^n x_n}{n}) = (x_1,\frac{x_1 + x_2}{2},\frac{x_1 + x_2 + x_3}{3}, ...)\rightarrow x$ \qedsymbol
\end{center}

{\Large \textbf{b.}} Let $(x_n) = ((-1)^n)$. We have shown many times $(x_n)$ does not converge.
\begin{center}
    \doublespacing
    Consider $(y_n) = (\frac{\sum _{i=1}^n x_n}{n}) = (x_1,\frac{x_1 + x_2}{2},\frac{x_1 + x_2 + x_3}{3}, ...) = (-1,\frac{-1 + 1}{2},\frac{-1 + 1 -1}{3}, ...) = (-1, 0, -\frac{1}{3}, 0, -\frac{1}{5}, ...)$
    \\If $n$ is even then $y_n = 0$ and if $n$ is odd then $y_n = -\frac{1}{n}$. Here $(y_n)\rightarrow 0$.
    \\Proof:
    \\Let $\epsilon > 0$ then by the Archimedean property there exists an $N\in\mathbb{N}$ such that $\frac{1}{N} <\epsilon$.
    \\So for $n\geq N$ if $n$ is even then $|y_n - 0| = |0 - 0| = 0 <\epsilon$ and if $n$ is odd then $|y_n - 0| = |-\frac{1}{n} - 0| = \frac{1}{n} <\epsilon$.
    \\This was for arbitrary $\epsilon > 0$ and is therefore true for all $\epsilon > 0$.
    \\So for all $\epsilon > 0$ there exists an $N\in\mathbb{N}$ such that if $n\geq N$ then $|y_n - 0| <\epsilon$.
    \\Therefore for $(x_n) = ((-1)^n)$ we have that $(y_n)\rightarrow 0$.
    \\So this is such an example of a sequence $(x_n)$ that does not converge where $(y_n)$ does.
\end{center}


\newpage
\section*{2.4.3}

{\Large \textbf{a.}} Let $(a_n) = (\sqrt{2}, \sqrt{2 + \sqrt{2}}, \sqrt{2 + \sqrt{2 + \sqrt{2}}}, ...)$. Then $a_{n+1} = \sqrt{2 + a_n}$ for all $n\in\mathbb{N}$.
\begin{center}
    \doublespacing
    \begin{itemize}
        \item Proving $(a_n)$ is monotonically increasing:
        \\Let $S = \{n\in\mathbb{N} : a_n < a_{n+1}\}$. We know $a_1 = \sqrt{2} < \sqrt{2 + \sqrt{2}} = a_2$. So $1\in S$.
        \\Now assume that $n\in S$, that is assume $a_{n} < a_{n+1}$. Then $2 + a_n < 2 + a_{n+1}$ and $\sqrt{2 + a_n} <\sqrt{2 + a_{n+1}}$. 
        \\Since $a_{m+1} = \sqrt{2 + a_m}$ we have that $a_{n+1} < a_{n+2}$. So $n + 1\in S$.
        \\Therefore since $1\in S$ and if $n\in S$ then $n + 1\in S$ we have that $S =\mathbb{N}$.
        \\So $a_n < a_{n+1}$ for all $n\in\mathbb{N}$ and therefore $(a_n)$ is monotonically increasing.
        \item Proving $(a_n)$ is bounded:
        \\$a_n > 0$ for all $n\in\mathbb{N}$ since $a_1 =\sqrt{2} > 0$ and $(a_n)$ is monotonically increasing. So $|a_n| = a_n$ for all $n\in\mathbb{N}$.
        \\Let $S = \{n\in\mathbb{N} : a_n < 2\}$. We know $a_1 = \sqrt{2} < 2$. So $1\in S$.
        \\Now assume that $n\in S$, that is assume $a_{n} < 2$. Then $2 + a_n < 2 + 2 = 4$ and $\sqrt{2 + a_n} <\sqrt{4} = 2$. 
        \\Since $a_{m+1} = \sqrt{2 + a_m}$ we have that $a_{n+1} < 2$. So $n + 1\in S$.
        \\Therefore since $1\in S$ and if $n\in S$ then $n + 1\in S$ we have that $S =\mathbb{N}$.
        \\So $|a_n| = a_n < 2$ for all $n\in\mathbb{N}$ and therefore $(a_n)$ is bounded.
        \item The limit of $(a_n)$:
        \\Since $(a_n)$ is monotonically increasing and bounded, by the monotone convergence theorem $(a_n)$ converges.
        \\The limit of $(a_n)$ should satisfy the equation $a =\sqrt{2 + a}$. So $a^2 = a + 2$ and $a^2 - a - 2 = (a - 2)(a + 1) = 0$.
        \\The limit can not be 1 since $a_1 = \sqrt{2} > 1$ and $(a_n)$ is monotonically increasing so $|a_n - 1|$ is increasing.
        \\So $(a_n)\rightarrow 2$.
    \end{itemize}
\end{center}


\newpage
\section*{2.4.5}

{\Large \textbf{a.}} Define $(x_n)$ by $x_1 = 2$ and $x_{n+1} = \frac{1}{2} (x_n + \frac{2}{x_n})$ for $n\in\mathbb{N}$.
\begin{center}
    \doublespacing
    \begin{itemize}
        \item Showing $x_n^2\geq 2$ for all $n\in\mathbb{N}$:
        \\We know {\large $x_n^2 = (\frac{1}{2} (x_{n-1} +\frac{2}{x_{n-1}}))^2 =\frac{1}{4} (x_{n-1}^2 + 4 +\frac{4}{x_{n-1}^2}) =\frac{x_{n-1}^2}{4} + 1 +\frac{1}{x_{n-1}^2}$} for $n\in\mathbb{N}$.
        \\$x_n > 0$ for all $n\in\mathbb{N}$ because $x_1 = 2 > 0$, and if $x_n > 0$ then $\frac{2}{x_n} > 0$ and so $\frac{1}{2} (x_n + \frac{2}{x_n}) = x_{n+1} > 0$.
        \\So $x_n > 0$ for all $n\in\mathbb{N}$ and therefore $\frac{1}{x_n}$ exists for all $n\in\mathbb{N}$.
        \\Now consider $0\leq (\frac{x_{n-1}}{2} -\frac{1}{x_{n-1}})^2 = \frac{x_{n-1}^2}{4} - 1 +\frac{1}{x_{n-1}^2} = \frac{x_{n-1}^2}{4} + 1 +\frac{1}{x_{n-1}^2} - 2 = x_{n}^2 - 2$.
        \\So $x_{n}^2 - 2\geq 0$ is always true and therefore $x_{n}^2\geq 2$ for all $n\in\mathbb{N}$.
        \item Showing $x_n - x_{n+1}\geq 0$ for all $n\in\mathbb{N}$:
        \\Consider {\large $x_n - x_{n+1} = x_n - \frac{1}{2} (x_n + \frac{2}{x_n}) = \frac{x_n}{2} -\frac{1}{x_n} =\frac{x_n^2 - 2}{2 x_n}$}.
        \\From the results above we know $x_{n}^2 - 2\geq 0$ and $x_n > 0$ for all $n\in\mathbb{N}$.
        \\Therefore $x_n - x_{n+1} = \frac{x_n^2 - 2}{2 x_n}\geq 0$ for all $n\in\mathbb{N}$.
        \item This shows $(x_n)$ is bounded and monotonically decreasing so $(x_n)$ converges by the monotone convergence theorem.
        \\The limit of $(x_n)$ should satisfy the equation $x =\frac{1}{2} (x +\frac{2}{x})$. So $2x = x +\frac{2}{x}$, and $x =\frac{2}{x}$.
        \\So $x^2 = 2$ and therefore $x =\sqrt{2}$. So $(x_n)\rightarrow\sqrt{2}$.
    \end{itemize}
\end{center}

{\Large \textbf{b.}} We want to modify $(x_n)$ so that $(x_n)\rightarrow\sqrt{c}$.
\begin{center}
    \doublespacing
    We should focus on the recursive definition to set the limit, $x =\frac{1}{2} (x +\frac{2}{x})$.
    \\If we change the recursive definition to $x_{n+1} =\frac{1}{2} (x_n +\frac{c}{x_n})$ we get the limit condition $x =\frac{1}{2} (x +\frac{c}{x})$. So $2x = x +\frac{c}{x}$. 
    \\We have $x =\frac{c}{x}$ and so $x^2 = c$. Therefore $x =\sqrt{c}$ so by this definition $(x_n)\rightarrow\sqrt{c}$.
    \\To make it look nicer you can also let $x_1 = c$ and the limit will not change.
\end{center}


\newpage
\section*{2.4.7}
\begin{center}
    Let $(a_n)$ be a bounded sequence. Then there exists an $M\in\mathbb{N}$ such that $|a_n|\leq M$ for all $n\in\mathbb{N}$.
\end{center}

{\Large \textbf{a.}} Let $(y_n) = (sup\{a_k : k\geq n\})$. Then I will show $(y_n)$ is monotonically decreasing.
\begin{center}
    \doublespacing
    \begin{itemize}
        \item Showing $(y_n)$ is monotonically decreasing:
        \\ Let $A_n = \{a_k : k\geq n\}$. Then $A_1\supseteq A_2\supseteq ...\supseteq A_n\supseteq ...$ and as proved in Sample Work 3, $sup A_1\geq sup A_2\geq ... sup A_n\geq ...$
        \\Since $y_n = sup A_n$ we have that $y_1\geq y_2\geq ...\geq y_n\geq ...$ so $(y_n)$ is monotonically decreasing.
        \item Showing $(y_n)$ is bounded:
        \\Since $(a_n)$ is bounded we have that $|a_n|\leq M$ for some $M\in\mathbb{N}$ and all $n\in\mathbb{N}$. So $-M\leq a_n\leq M$ for all $n\in\mathbb{N}$.
        \\Since $a_n\leq M$ for all $n\in\mathbb{N}$ we have that $M$ is an upper bound of each $A_n$. So $sup A_n\leq M$ for all $n\in\mathbb{N}$.
        \\Furthermore for any $A_n$ we have that for all $a_k\in A_n$, $-M\leq a_k\leq sup A_n\leq M$.
        \\So $|y_n| = |sup A_n|\leq M$ for all $n\in\mathbb{N}$ and $(y_n)$ is bounded.
    \end{itemize}
    Therefore since $(y_n)$ is monotonically decreasing and bounded $(y_n)$ converges by the monotone convergence theorem \qedsymbol
\end{center}

{\Large \textbf{b.}} Let $(x_n) = (inf\{a_k : k\geq n\})$. Since $(a_n)$ is bounded the infimum exists for each of these sets.
\begin{center}
    \doublespacing
    \begin{itemize}
        \item Proving $inf A\leq inf B$ for two sets $A, B$ such that $A\supseteq B$:
        \\Let $A$ and $B$ both be nonempty sets so $A\supseteq B$.
        \\Let $x = inf A$. Then $x\leq a$ for all $a\in A$. Since $A\supseteq B$ if $b\in B$ then $b\in A$.
        \\So we also know $x\leq b$ for all $b\in B$ and is a lower bound of $B$, meaning it must be less than or equal to $inf B$.
        \\Therefore $x = inf A\leq inf B$
        \item Showing $(x_n)$ is monotonically increasing:
        \\Let $A_n = \{a_k : k\geq n\}$. Then $A_1\supseteq A_2\supseteq ...\supseteq A_n\supseteq ...$ and from above $inf A_1\leq inf A_2\leq ...\leq A_n\leq ...$
        \\Since $x_n = inf A_n$ we have that $x_1\leq x_2\leq ...\leq x_n\leq ...$ so $(x_n)$ is monotonically increasing.
        \item Showing $(x_n)$ is bounded:
        \\Since $(a_n)$ is bounded we have that $|a_n|\leq M$ for some $M\in\mathbb{N}$ and all $n\in\mathbb{N}$. So $-M\leq a_n\leq M$ for all $n\in\mathbb{N}$.
        \\Since $-M\leq a_n$ for all $n\in\mathbb{N}$ we have that $-M$ is a lower bound of each $A_n$. So $inf A_n\geq -M$ for all $n\in\mathbb{N}$.
        \\Furthermore for any $A_n$ we have that for all $a_k\in A_n$, $-M\leq inf A_n\leq a_k\leq M$.
        \\So $|x_n| = |inf A_n|\leq M$ for all $n\in\mathbb{N}$ and $(x_n)$ is bounded.
    \end{itemize}
    Therefore since $(x_n)$ is monotonically increasing and bounded $(x_n)$ converges by the monotone convergence theorem \qedsymbol
\end{center}

{\Large \textbf{c.}} For any $A_n$ we have shown in the previous two parts that $-M\leq inf A_n\leq a_k\leq sup A_n\leq M$.
\begin{center}
    \doublespacing
    So we have that $x_n = inf A_n\leq sup A_n = y_n$ for all $n\in\mathbb{N}$.
    \\Therefore since $(x_n)$ and $(y_n)$ converge, say to $x$ and $y$ respectively, we may use the order limit theorem to say $x\leq y$.
    \\Since $lim\;\: inf\;a_n = x$ and $lim\;\: sup\;a_n = y$ we have shown that $lim\;\: inf\;a_n\leq lim\;\: sup\;a_n$ for any bounded sequence $(a_n)$.
\end{center}

\newpage
{\Large \textbf{d.}}
\begin{center}
    \doublespacing
    \begin{itemize}
        \item Showing if $lim\;\: inf\;a_n = lim\;\: sup\;a_n = a$ then $(a_n)\rightarrow a$:
        \\Assume $lim\; x_n = lim\;\: inf\;a_n = lim\;\: sup\;a_n = lim\; y_n = a$. Let $\epsilon > 0$ and $\alpha =\epsilon / 2$.
        \\Then there exists an $N\in\mathbb{N}$ such that $|x_n - a| <\alpha$ and $|y_n - a| <\alpha$ for $n\geq N$.
        \\Since $x_n\leq a_n\leq y_n$ we have $0\leq a_n - x_n\leq y_n - x_n$.
        \\So $|a_n - x_n|\leq |y_n - x_n| = |(y_n - a) + (a - x_n)|\leq |y_n - a| + |x_n - a| < 2\alpha =\epsilon$ for $n\geq N$.
        \\Therefore if $lim\;\: inf\;a_n = lim\;\: sup\;a_n = a$ then $(a_n)\rightarrow a$.
        \item Showing if $(a_n)\rightarrow a$ then $lim\;\: inf\;a_n = lim\;\: sup\;a_n = a$:
        \\Assume $(a_n)\rightarrow a$. Let $\epsilon > 0$. Then there exists an $N\in\mathbb{N}$ such that $|a_n - a| <\epsilon$ for $n\geq N$.
        \\So $-\epsilon < a_n - a <\epsilon$ and $a -\epsilon < a_n < a +\epsilon$ for $n\geq N$.
        \\Therefore $a -\epsilon$ is a lower bound for $A_N$ and $a +\epsilon$ is an upper bound for $A_N$.
        \\Since for $n > N$, $A_n\subseteq A_N$ we have that if $b\in A_n$ then $b\in A_N$ so $a -\epsilon < b < a +\epsilon$.
        \\So $a -\epsilon$ is a lower bound for $A_n$ and $a +\epsilon$ is an upper bound for $A_n$ when $n\geq N$.
        \\So we have $a -\epsilon\leq inf\;a_n\leq sup\;a_n\leq a +\epsilon$ for $n\geq N$.
        \\Therefore $|inf\;a_n - a| <\epsilon$ and $|sup\;a_n - a| <\epsilon$ for $n\geq N$.
        \\So if $(a_n)\rightarrow a$ then $lim\;\: inf\;a_n = lim\;\: sup\;a_n = a$.
    \end{itemize}
    Therefore $(a_n)\rightarrow a$ and $lim\;a_n$ exists if and only if $lim\;\: inf\;a_n = lim\;\: sup\;a_n = a$ \qedsymbol
\end{center}


\section*{Extra Problem}
\begin{center}
    \doublespacing
    Let $(x_n) = (\frac{n^2 + 1}{2n + 1})$ for $n\in\mathbb{N}$.
    \\Then {\large $\frac{n^2 + 1}{2n + 1} >\frac{n^2 + 1 - 2}{2n + 1} =\frac{n^2 - 1}{2n + 1} =\frac{(n + 1)(n - 1)}{2n + 1} >\frac{(n + 1)(n - 1)}{2n + 1 + 1} =\frac{(n + 1)(n - 1)}{2n + 2} =\frac{(n + 1)(n - 1)}{2(n + 1)} =\frac{n - 1}{2}$}
    \\Clearly $(y_n) = (\frac{n - 1}{2})$ diverges to $\infty$ as $\mathbb{N}$ is unbounded.
    \\Proof:
    \\$(y_n)$ is monotonically increasing since $y_{n+1} - y_n =\frac{(n + 1) - 1}{2} - \frac{n - 1}{2} =\frac{n - (n - 1)}{2} =\frac{1}{2} > 0$.
    \\So $y_{n+1} - y_n > 0$ for all $n\in\mathbb{N}$ and therefore $y_{n+1} > y_n$ for all $n\in\mathbb{N}$.
    \\Since $y_1 = 0$ and $(y_n)$ is monotonically increasing we have that $y_n\geq 0$ for all $n\in\mathbb{N}$.
    \\Assume for the sake of contradiction that $(y_n)$ is bounded, that is assume $|y_n|\leq M$ for some $M\in\mathbb{N}$.
    \\Then $|y_n| = y_n =\frac{n - 1}{2}\leq M$ for all $n\in\mathbb{N}$.
    \\But consider $y_{2M + 2} =\frac{(2M + 2) - 1}{2} =\frac{2M + 1}{2} = M +\frac{1}{2} > M$
    \\So we have a contradiction and therefore $(y_n)$ can not be bounded.
    \\Since $(y_n)$ is unbounded it can not converge since all convergent series are bounded.
    \\So $(y_n)$ diverges, is monotonically increasing, non-negative, and unbounded, and consequently $y_n$ diverges to $\infty$.
    \\Therefore since $x_n =\frac{n^2 + 1}{2n + 1} >\frac{n - 1}{2} = y_n$ and $(y_n)$ diverges to $\infty$ we have that $(x_n)$ diverges to $\infty$ \qedsymbol
\end{center}

\end{document}
