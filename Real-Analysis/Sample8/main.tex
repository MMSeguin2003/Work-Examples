\documentclass{article}
\usepackage{graphicx} % Required for inserting images
\usepackage[utf8]{inputenc}
\usepackage{setspace}
\usepackage[margin=1.5cm]{geometry}
\usepackage{amsmath}
\usepackage{amsthm}
\usepackage{amsfonts}
\usepackage{indentfirst}

\title{Continuity and Cauchy Sequences}
\author{Matthew Seguin}
\date{}

\begin{document}

\maketitle


\section*{4.3.9}
\begin{center}
    \doublespacing
    Let $h:\mathbb{R}\rightarrow\mathbb{R}$ be continuous on $\mathbb{R}$ and let $K =\{x : h(x) = 0\}$.
    \\If $K$ has no limit points then it is trivially closed.
    \\Otherwise let $x$ be a limit point of $K$. Then there exists some sequence $(x_n)$ contained in $K$ such that $(x_n)\rightarrow x$.
    \\Since $x_n\in K$ for all $n\in\mathbb{N}$ it must be that $h(x_n) = 0$ for all $n\in\mathbb{N}$.
    \\So $(h(x_n)) = (0)$ which clearly converges to $0$.
    \\Since $h$ is continuous on $\mathbb{R}$ it must be that $lim\;h(x_n) = h(x)$. So we have that $h(x) = 0$ and therefore $x\in K$.
    \\This was for an arbitrary limit point of $K$ and is therefore true for all limit points of $K$.
    \\So $K$ contains all its limit points and is therefore closed \qedsymbol
\end{center}


\newpage
\section*{4.3.11}
\begin{center}
    Let $f:\mathbb{R}\rightarrow\mathbb{R}$ such that there exists a $c\in (0, 1)$ where $|f(x) - f(y)|\leq c|x - y|$ for all $x, y\in\mathbb{R}$.
\end{center}

{\Large\textbf{a.}} Let $a\in\mathbb{R}$, let $\epsilon > 0$, and let $\delta = \epsilon / c$. Then $\delta > 0$ since $c > 0$.
\begin{center}
    \doublespacing
    Then if $|x - a| <\delta$ we have that $|f(x) - f(a)|\leq c|x - a| < c\delta =\epsilon$.
    \\This was for arbitrary $\epsilon > 0$ and is therefore true for all $\epsilon > 0$.
    \\So for all $\epsilon > 0$ we have found a $\delta > 0$ such that when $|x - a| <\delta$ it follows that $|f(x) - f(a)| <\epsilon$.
    \\So $f$ is continuous at $a$, and this was for arbitrary $a\in\mathbb{R}$ and is therefore true for all $a\in\mathbb{R}$.
    \\Therefore $f$ is continuous on $\mathbb{R}$ \qedsymbol
\end{center}

{\Large\textbf{b.}} Let $y_1\in\mathbb{R}$ and define $(y_n)$ by $y_{n+1} = f(y_n)$.
\begin{center}
    \doublespacing
    If $(y_n)$ is a constant sequence it is trivially convergent and therefore Cauchy.
    \\Otherwise since $y_{n+1} = f(y_n)\in\mathbb{R}$ we have that $|y_{n+2} - y_{n+1}| = |f(y_{n+1}) - f(y_n)|\leq c|y_{n+1} - y_n|$ for all $n\in\mathbb{N}$.
    \\Furthermore $|y_{n+2} - y_{n+1}|\leq c|y_{n+1} - y_n| = c|f(y_n) - f(y_{n-1})|\leq c^2 |y_n - y_{n-1}| = ...\leq c^n |y_2 - y_1|$.
    \\Let $m, n\in\mathbb{N}$ where $m > n$.
    \\Then $|y_m - y_n| = |y_m - y_{m-1} + y_{m-1} - ... - y_{n+1} + y_{n+1} - y_n|\leq |y_m - y_{m-1}| + |y_{m-1} - y_{m-2}| + ... + |y_{n+1} - y_n|$.
    \\So $|y_m - y_n|\leq |y_m - y_{m-1}| + |y_{m-1} - y_{m-2}| + ... + |y_{n+1} - y_n|\leq c^{m-2} |y_2 - y_1| + ... + c^{n-1} |y_2 - y_1|$.
    \\Now we have $|y_m - y_n|\leq c^{m-2} |y_2 - y_1| + ... + c^{n-1} |y_2 - y_1| = c^{n-1} |y_2 - y_1| (1 + c + c^2 + ... + c^{m-n-1})$.
    \\Since $c > 0$ we have $|y_m - y_n|\leq c^{n-1} |y_2 - y_1| (1 + c + c^2 + ... + c^{m-n-1}) < c^{n-1} |y_2 - y_1| (1 + c + c^2 + ...) = \frac{c^{n-1} |y_2 - y_1|}{1 - c}$.
    \\Let $\epsilon > 0$ then let $N > log(\frac{\epsilon (1 - c)}{|y_2 - y_1|}) / log\;c + 1$. Such an $N$ exists because $c, y_2, y_1,\;and\;\:\epsilon$ are all fixed and $c\in (0, 1)$ and $y_1\neq y_2$ because our assumption was that $(y_n)$ was not constant.
    \\Then $(N - 1)\:log\;c < log(\frac{\epsilon (1 - c)}{|y_2 - y_1|})$ and $c^{N-1} < \frac{\epsilon (1 - c)}{|y_2 - y_1|}$ and $\frac{|y_2 - y_1| c^{N-1}}{1-c} <\epsilon$.
    \\So for $m, n\geq N$ we have $|y_m - y_n| < \frac{c^{n-1} |y_2 - y_1|}{1 - c}\leq \frac{c^{N-1} |y_2 - y_1|}{1 - c} <\epsilon$.
    \\This was for arbitrary $\epsilon > 0$ and is therefore true for all $\epsilon > 0$.
    \\So for all $\epsilon > 0$ there exists an $N\in\mathbb{N}$ such that if $m, n\geq N$ it follows that $|y_m - y_n| <\epsilon$.
    \\Therefore $(y_n)$ is Cauchy and consequently is convergent. So we can say $lim\; y_n = y$ for some $y\in\mathbb{R}$ \qedsymbol
\end{center}

{\Large\textbf{c.}} Say $lim\; y_n = y$, then since $f$ is continuous as per part a we know $f(y) = lim\; f(y_n) = lim\; y_{n+1} = lim\; y_n = y$.
\begin{center}
    \doublespacing
    So $y$ is a fixed point of $f$.
    \\Let $x$ be a fixed point of $f$. Then $|x - y| = |f(x) - f(y)|$ and simultaneously $|f(x) - f(y)|\leq c|x - y|$.
    \\It can not be that $|f(x) - f(y)| < c|x - y|$ since $0 < c < 1$ and $|f(x) - f(y)| = |x - y|$.
    \\So we have that $|f(x) - f(y)| = |x - y|$ and $|f(x) - f(y)| = c|x - y|$.
    \\Consequently, $c|x - y| = |x - y|$ and $(1 - c)|x - y| = 0$. So $|x - y| = 0$ and therefore $x = y$ so $y$ is the only fixed point \qedsymbol
\end{center}

{\Large\textbf{d.}} Let $x\in\mathbb{R}$ then consider the sequence $(x, f(x), f(f(x)), ...)$.
\begin{center}
    \doublespacing
    Without loss of generality we can say that this sequence converges and it converges to a fixed point.
    \\Since $y$ is the only fixed point it must be that $(x, f(x), f(f(x)), ...)\rightarrow y$.
    \\This was for arbitrary $x\in\mathbb{R}$ and is therefore true for all $x\in\mathbb{R}$.
    \\So the sequence $(x, f(x), f(f(x)), ...)\rightarrow y$ for all $x\in\mathbb{R}$ \qedsymbol
\end{center}


\newpage
\section*{4.4.6}

{\Large\textbf{a.}} Let $f(x) =\frac{1}{x}$ for $x\in (0, 1)$, and let $(x_n) = (\frac{1}{n})$.
\begin{center}
    \doublespacing
    The functions $g(x) = x$ and $h(x) = 1$ are continuous on $\mathbb{R}$ and therefore also on $(0, 1)$ as shown below.
    \\Let $c\in\mathbb{R}$ and let $\epsilon > 0$, then let $\delta =\epsilon$.
    \\Then if $|x - c| <\delta$ we have $|g(x) - g(c)| = |x - c| <\delta =\epsilon$ and $|h(x) - h(c)| = |1 - 1| = 0 <\epsilon$.
    \\This was for arbitrary $c\in\mathbb{R}$ and is consequently true for all $c\in\mathbb{R}$ so $g$ and $h$ are continuous on $\mathbb{R}$.
    \\So by the algebraic continuity theorem $f(x) =\frac{1}{x} =\frac{h(x)}{g(x)}$ is continuous on $(0, 1)$ since the quotient is defined.
    \\Then as seen many times previously $(x_n) = (\frac{1}{n})\rightarrow 0$ so $(x_n)$ is Cauchy.
    \\But $(f(x_n)) = (\frac{1}{1/n}) = (n)$ is unbounded and therefore can not be Cauchy.
    \\This is such an example of a Cauchy sequence $(x_n)$ and a continuous function $f: (0, 1)\rightarrow\mathbb{R}$ where $f(x_n)$ is not Cauchy.
\end{center}

{\Large\textbf{b.}} This is not possible. Let $f: (0, 1)\rightarrow\mathbb{R}$ be uniformly continuous and let $(x_n)\subset (0, 1)$ be Cauchy.
\begin{center}
    \doublespacing
    Then for all $\epsilon > 0$ there exists a $\delta > 0$ such that if $|x - y| <\delta$ then $|f(x) - f(y)| <\epsilon$.
    \\Since $(x_n)$ is Cauchy for any $\alpha > 0$ there exists an $N\in\mathbb{N}$ such that if $m, n\geq N$ then $|x_m - x_n| <\alpha$.
    \\So let $\epsilon > 0$, then there exists a $\delta > 0$ so that if $|x - y| <\delta$ then $|f(x) - f(y)| <\epsilon$.
    \\Furthermore there exists an $N\in\mathbb{N}$ such that if $m, n\geq N$ then $|x_m - x_n| <\delta$.
    \\So for all $m, n\geq N$ we have $|f(x_m) - f(x_n)| <\epsilon$.
    \\Therefore for all $\epsilon > 0$ we have found an $N\in\mathbb{N}$ such that if $m, n\geq N$ then $|f(x_m) - f(x_n)| <\epsilon$.
    \\So $f(x_n)$ is Cauchy \qedsymbol
\end{center}

{\Large\textbf{c.}} This is not possible. Let $f: [0,\infty )\rightarrow\mathbb{R}$ be continuous and let $(x_n)\subset [0,\infty )$ be Cauchy.
\begin{center}
    \doublespacing
    Then $(x_n)\rightarrow x$ for some $x\in\mathbb{R}$ and $x$ is therefore a limit point of $[0,\infty )$.
    \\Since $[0,\infty )$ is closed we have that $x\in [0,\infty )$.
    \\So $f(x)$ is defined and since $f$ is continuous we have $(f(x_n))\rightarrow f(x)$.
    \\Therefore $(f(x_n))$ is Cauchy \qedsymbol
\end{center}


\newpage
\section*{4.4.9}

{\Large\textbf{a.}} Assume $f: A\rightarrow\mathbb{R}$ is Lipschitz, then there exists an $M > 0$ such that $|\frac{f(x) - f(y)}{x - y}|\leq M$ for all $x, y\in A$ where $x\neq y$.
\begin{center}
    \doublespacing
    Let $\epsilon > 0$ then let $\delta =\epsilon / M$. Let $x, y\in A$.
    \\If $x = y$ then $|x - y| = 0 <\delta$ and $|f(x) - f(y)| = 0 <\epsilon$.
    \\Otherwise if $|x - y| <\delta =\epsilon / M$ we have $|f(x) - f(y)|\leq M|x - y| < M\delta =\epsilon$.
    \\This was for arbitrary $x, y\in A$ and is therefore true for all $x, y\in A$.
    \\So for all $x, y\in A$ and for all $\epsilon > 0$ we have found a $\delta > 0$ such that if $|x - y| <\delta$ then $|f(x) - f(y)| <\epsilon$.
    \\So $f$ is uniformly continuous \qedsymbol
\end{center}

{\Large\textbf{b.}} No, the converse is not true. Let $f(x) =\sqrt{x}$ which is uniformly continuous on $[0,\infty )$ as seen in class.
\begin{center}
    \doublespacing
    However, $f$ is not Lipschitz:
    \\Let $y = 0$ and $x\neq 0$, then $|\frac{f(x) - f(y)}{x - y}| = |\frac{\sqrt{x} - 0}{x - 0}| = |\frac{\sqrt{x}}{x}| =\frac{1}{\sqrt{x}}$.
    \\Also $lim _{x\rightarrow 0^+} |\frac{f(x) - f(y)}{x - y}| = lim _{x\rightarrow 0^+} \frac{1}{\sqrt{x}} =\infty$.
    \\Let $M > 0$ then let $\delta = \frac{1}{M^2}$. Then if $0 < x <\delta =\frac{1}{M^2}$ we have $\sqrt{x} <\sqrt{\frac{1}{M^2}} =\frac{1}{M}$ and $\frac{1}{\sqrt{x}} > M$.
    \\This was for arbitrary $M > 0$ and is therefore true for all $M > 0$ so $lim _{x\rightarrow 0^+}\frac{1}{\sqrt{x}} =\infty$.
    \\So $|\frac{f(x) - f(y)}{x - y}|$ is unbounded for $y = 0$ and $x\rightarrow 0$.
    \\Therefore $f(x) =\sqrt{x}$ is not Lipschitz but is uniformly continuous.
\end{center}


\newpage
\section*{4.5.2}

{\Large\textbf{a.}} Let $f(x) = |x|$ if $x\in [-1, 1]$ and $f(x) = 1$ if $x\in (-2, -1)\cup (1, 2)$.
\begin{center}
    \doublespacing
    Since $|x|$ is continuous on $\mathbb{R}$ we have $f$ is continuous on $(-1, 1)$.
    \\Since constant functions are continuous on $\mathbb{R}$ we have $f$ is continuous on $(-2, -1)\cup (1, 2)$.
    \\Since $lim _{x\rightarrow {-1}^+} f(x) = lim _{x\rightarrow {-1}^+} |x| = 1 = lim_{x\rightarrow {-1}^-} 1 = lim_{x\rightarrow {-1}^-} f(x) = f(-1)$ we have that $f$ is continuous at $-1$.
    \\Since $lim _{x\rightarrow {1}^-} f(x) = lim _{x\rightarrow {1}^-} |x| = 1 = lim_{x\rightarrow {1}^+} 1 = lim_{x\rightarrow {1}^+} f(x) = f(1)$ we have that $f$ is continuous at $1$.
    \\Let $I = (-2, 2)$ then $f: I\rightarrow\mathbb{R}$ is continuous, and clearly $f(I) = [0, 1]$.
    \\Clearly $(-2, 2)$ is an open interval and $[0, 1]$ is a closed interval.
    \\So this is such an example of a continuous function whose domain is an open interval and whose range is a closed interval.
\end{center}


{\Large\textbf{b.}} This is not possible. Let $f: [a, b]\rightarrow\mathbb{R}$ be continuous.
\begin{center}
    \doublespacing
    Then since $[a, b]$ is closed and bounded it is compact.
    \\So since $f$ is continuous the range of $f$ is compact and is therefore closed and bounded.
    \\Therefore the range of $f$ can not be open because the only sets that are open and closed are $\phi$ and $\mathbb{R}$.
    \\But the range of $f$ is not empty since its domain is nonempty and the range of $f$ is not $\mathbb{R}$ since it is bounded.
    \\Therefore since the range of $f$ is already closed and it is neither $\phi$ or $\mathbb{R}$, the range of $f$ is not open \qedsymbol
\end{center}

{\Large\textbf{c.}} Let $f(x) =\frac{1}{x}$ for $x\in(0,1]$ and $f(x) = 1$ for $x\in (1,2)$.
\begin{center}
    \doublespacing
    Since $\frac{1}{x}$ is continuous on $(0,\infty )$ we have $f$ is continuous on $(0, 1)$.
    \\Since constant functions are continuous on $\mathbb{R}$ we have $f$ is continuous on $(1, 2)$.
    \\Since $lim _{x\rightarrow {1}^-} f(x) = lim _{x\rightarrow {1}^-}\frac{1}{x} = 1 = lim_{x\rightarrow {1}^+} 1 = lim_{x\rightarrow {1}^+} f(x) = f(1)$ we have that $f$ is continuous at $1$.
    \\Let $I = (0, 2)$ then $f: I\rightarrow\mathbb{R}$ is continuous and clearly $f(I) = [1, \infty )$.
    \\Clearly $(0, 2)$ is an open interval and $[1,\infty )$ is an unbounded, closed set.
    \\So this is such an example of a continuous function whose domain is an open interval and whose range is an unbounded, closed set that is not $\mathbb{R}$.
\end{center}

{\Large\textbf{d.}} This is not possible. Let $f:\mathbb{R}\rightarrow\mathbb{R}$ be continuous. Let $\mathbb{I}$ denote the set of irrational numbers.
\begin{center}
    \doublespacing
    If $f$ is constant we are done since then $f(\mathbb{R}) =\{c\}\neq\mathbb{Q}$.
    \\Otherwise consider $f$ over an interval $[a, b]$ where $f(a)\neq f(b)$, this can be done because $f$ is non-constant here.
    \\Then there must exist some $y\in\mathbb{I}$ such that $f(a) < y < f(b)$ since $\mathbb{I}$ is dense in $\mathbb{R}$.
    \\Since $f$ is continuous on $[a, b]$ by the intermediate value theorem there exists some $x\in (a, b)$ such that $f(x) = y\notin\mathbb{Q}$.
    \\Therefore $f(\mathbb{R})\neq\mathbb{Q}$ \qedsymbol
\end{center}


\newpage
\section*{4.5.8}
\begin{center}
    \doublespacing
    Let $f:[a, b]\rightarrow\mathbb{R}$ be continuous and one to one. Then $f^{-1}: f([a, b])\rightarrow [a, b]$ exists.
    \\Assume $f^{-1}: f([a, b])\rightarrow [a, b]$ is not continuous for the sake of contradiction.
    \\Then there exists some $(x_n)\subset f([a, b])$ such that $(x_n)\rightarrow x$ but $(f^{-1} (x_n))\not\rightarrow f^{-1}(x)$.
    \\Note that $[a, b]$ is compact and since $f$ is continuous $f([a, b])$ is compact and is consequently closed so $x\in f([a, b])$.
    \\Therefore $f(y) = x$ for some $y\in [a, b]$. So our above statement amounts to $(x_n)\rightarrow x$ but $(f^{-1} (x_n))\not\rightarrow y$.
    \\Let $(y_n) = (f^{-1} (x_n))$. Then $(y_n)\not\rightarrow y$ by assumption.
    \\However since $f^{-1}: f([a, b])\rightarrow [a, b]$ we have that $(y_n)\subset [a, b]$ and therefore $(y_n)$ is bounded.
    \\Therefore there exists a subsequence $(y_{n_k})$ that converges to some $y_0\neq y$.
    \\Since $f$ is continuous it must be that $(f(y_{n_k}))\rightarrow f(y_0)$ but $(f(y_{n_k})) = (f(f^{-1} (x_{n_k}))) = (x_{n_k})\rightarrow x$
    \\This implies $f(y_0) = x$ and $f(y) = x$ but $y_0\neq y$, a contradiction since $f$ is one to one.
    \\Therefore it must be that $f^{-1}$ is continuous \qedsymbol
\end{center}

\end{document}
